% arXiv formatting
\documentclass[aos,preprint]{imsart}
\setattribute{journal}{name}{} 

\RequirePackage[OT1]{fontenc}
\RequirePackage{amsthm,amsmath,amssymb}

% ICML-style citing
\RequirePackage{natbib}
\setcitestyle{round,authoryear,citesep={;},aysep={,},yysep={;}}
\newcommand{\yrcite}[1]{\citeyearpar{#1}}
\renewcommand{\cite}[1]{\citep{#1}}

\RequirePackage[colorlinks,citecolor=blue,urlcolor=blue]{hyperref}

% use Times
%\usepackage{times}
% For figures
\usepackage{graphicx} % more modern

\usepackage{macros}

\usepackage{caption}
\usepackage{subcaption}

%\usepackage{amsmath,amssymb}
\usepackage{verbatim}
\usepackage{gensymb}

%\usepackage{fullpage}
\usepackage[top=1in, bottom=1in, left=1in, right=1in]{geometry}

% My Packages
\usepackage{macros}
\usepackage{savesym}
\savesymbol{captionbox}
\usepackage[font={small}]{caption}
\DeclareCaptionType{copyrightbox}
\usepackage{subcaption}
\restoresymbol{CAP}{captionbox}

\usepackage{amsmath,amssymb}
\usepackage{verbatim}
\usepackage{gensymb}
\usepackage{multirow}
\usepackage{todonotes}


\begin{document}

\begin{frontmatter}
\title{Sampling the P\'{o}lya-gamma Distribution with Small Shape Parameter}
\runtitle{Dependent Multinomial Models Made Easy}


\begin{aug}
\author{\fnms{Scott W.} \snm{Linderman}\ead[label=e1]{swl@seas.harvard.edu}}
\affiliation{Harvard University}
\runauthor{S. W. Linderman}
\end{aug}

%\maketitle
\begin{abstract}
  Efficient sampling of the P\'{o}lya-gamma
  distribution is critical to the success of
  the augmentation schemes introduced by \citet{polson2013bayesian}.
  To address this issue, \citet{windle2014sampling} develop a number
  of sampling schemes for
  the P\'{o}lya-gamma distribution, however they do not address the small
  ``shape'' parameter regime. Efficient sampling in this regime
  would enable the use of marginal estimation techniques like
  annealed importance sampling. Moreover, since the P\'{o}lya-gamma
  distribution is infinitely divisible, a sampler for the small
  shape regime would enable efficient simulation of a
  P\'{o}lya-gamma Levy process.
\end{abstract}
\end{frontmatter}

\section{Introduction}
The \polyagamma distribution,~$\distPolyaGamma(b)$, is parameterized
by a ``shape'' parameter~$b$. It is defined by its Laplace transform,
\begin{align}
  \cosh^{-b}(\sqrt{t/2}) = \int_0^\infty \exp\{-tx\} \distPolyaGamma(x \given b) \,\mathrm{d}x.
\end{align}
This is simply a scaling of the~$\distJacobi(b)$ density surveyed by \citet{biane2001probability}, and defined by Laplace transform
\begin{align}
  \cosh^{-b}(\sqrt{2t}) &=  \int_{0}^\infty \exp\{-tx\} \distJacobi(x \given b) \, \mathrm{d}x. 
\end{align}
Thus, we see that if~$X \sim \distPolyaGamma(b)$ and~$Y \sim \distJacobi(b)$, then~$X \disteq Y/4$. \citet{polson2013bayesian}
define exponentially tilted versions of these distributions as follows,
\begin{align}
  \distJacobi(x \given b, z) &= \cosh^b(z) e^{-xz^2/2} \, \distJacobi(x \given b), \\
  \distPolyaGamma(x \given b, z) &= \cosh^b(z/2) e^{-xz^2/2} \, \distPolyaGamma(x \given b) \\
  &= \tfrac{1}{4} \distJacobi(x \given b, z/2).
\end{align}

The~$\distJacobi(b)$ density can be written as an infinite alternating sum,
\begin{align}
  \distJacobi(x \given b) &= \frac{2^b}{\Gamma(b)}
  \sum_{n=0}^\infty (-1)^n \frac{\Gamma(n+b)}{\Gamma(n+1)} \frac{(2n+b)}{\sqrt{2\pi x^3}}
  \exp\left\{- \frac{(2n+b)^2}{2x} \right\}.
\end{align}


Following \citet{windle2014sampling}, we derive a sampling algorithm for
the~$\distJacobi(b,z)$ distribution and then scale the result to obtain
a sample from the corresponding \polyagamma distribution. We focus specifically
on the case where~$b<1$.

\section{Sampling~$\distJacobi(b)$ for~$b<1$}
Empirically, it appears that the \polyagamma density is largely determined
by the first three terms of thi sum when~$b<1$.
\todo[inline]{substantiate this with
  a plot of (earthmover?) distance between distributions}

Considering only the first three terms, we have
\begin{align}
  \distJacobi(x \given b) &\approx \frac{2^b}{\Gamma(b) \sqrt{2 \pi x^3}}
  \bigg[  b\Gamma(b)       \exp \left \{-\frac{b^2}{2x}      \right \} \\
\nonumber & \qquad \qquad \qquad -b\Gamma(b) (2+b) \exp \left \{ -\frac{(2+b)^2}{2x} \right \} \\
\nonumber & \qquad \qquad \qquad +b\Gamma(b)\frac{(1+b)}{2} (4+b) \exp \left \{ -\frac{(4+b)^2}{2x} \right \}
\bigg] \\
  &= \frac{b 2^b}{\sqrt{2 \pi x^3}} \exp \left \{-\frac{b^2}{2x} \right \}
  \bigg[1
        - (2+b) \exp \left \{-\frac{4+4b}{2x} \right \}
        + \frac{(1+b)(4+b)}{2}  \exp \left \{-\frac{16+8b}{2x} \right \}
  \bigg]
\end{align}

For small~$x$, the latter terms inside the square brackets are much less than one,
leading to the follow approximation,
\begin{align}
  \distJacobi(x \given b) &\approx \frac{b 2^b}{\sqrt{2 \pi x^3}} \exp \left \{-\frac{b^2}{2x} \right \} \quad \text{ when } x \ll 1.
\end{align}
Note that this is a
scaled inverse gamma distribution,
\begin{align}
  2^b \, \distInvGamma(x \given \tfrac{1}{2}, \tfrac{b^2}{2})
  &= 2^b \frac{\sqrt{b^2/2}}{\Gamma(\tfrac{1}{2})} x^{-\tfrac{3}{2}} \exp\left\{-\frac{b^2}{2x} \right\} \\
  &= \frac{b 2^b}{\sqrt{2 \pi x^3}} \exp\left\{-\frac{b^2}{2x} \right\}.
  %\\
  %&\approx \distJacobi(x \given b) \quad \text{ when } x \ll 1.
\end{align}

For larger~$x$, the second and third terms are nonnegligible. In particular, the sum of these terms is between zero and one, and leads to much lighter tails than those
produced by an inverse gamma distribution. Empirically, these tails
appear to be effectively exponential. That is, for~$x>1$ we can make the following approximation,
\begin{align}
  \distJacobi(x \given b) &\approx \exp \left \{c(b) + m(b) x \right \}
  \quad \text{ when } x>1,
\end{align}
for some functions~$c(b)$ and~$m(b)$ that do not depend on~$x$.
We will derive such functions by linearizing the log density.
We have,
\begin{align}
  \log \distJacobi(x \given b) &\approx \log \left( \frac{b 2^b}{\sqrt{2 \pi}} \right)
   -\frac{b^2}{2x} - \tfrac{3}{2} \log x
  +\log \left( 1
        - (2+b) \exp \left \{-\frac{4+4b}{2x} \right \}
        + \frac{(1+b)(4+b)}{2}  \exp \left \{-\frac{16+8b}{2x} \right \}
        \right)
\end{align}







    
\bibliography{draft}
\bibliographystyle{unsrtnat}

\end{document} 

